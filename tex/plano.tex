\chapter{Plano de Trabalho}
\label{cap:plano}

\section{Atividades}


O desenvolvimento deste trabalho de conclusão de curso está dividido em cinco atividades principais e na confecção de duas monografias, constando o presente documento como a primeira delas. A segunda monografia será confeccionada a partir desse documento juntamente com os resultados obtidos ao longo do segundo semestre (TCC-II).

\subsection{Revisão Bibliográfica}

Embora uma ampla revisão bibliográfica já tenha sido realizada para a confecção do presente documento, essa atividade estará em permanente execução com o objetivo de ampliar a revisão já elaborada além de atualizar as referências com modelos recentemente desenvolvidos e publicados. Mais especificamente, essa revisão se concentrará na busca por modelos de detecção de faces em vídeos e também por modelos de reconhecimento de faces. Alguns modelos serão selecionados e utilizados como parâmetros de comparação com o algoritmo desenvolvido.

\subsection{Implementação das técnicas selecionadas}

Como mencionado na seção anterior, alguns modelos encontrados na literatura serão selecionados e implementados. Essa etapa de implementação dos modelos selecionados caracteriza a segunda atividade desse trabalho de conclusão de curso. Como já mencionado no projeto inicial do TCC-I apresentado, todas as implementações serão realizadas em linguagem C/C++ visando a obtenção de um melhor desempenho computacional e possiblidade de aplicação do sistema em tempo real.

\subsection{Testes do sistema}

Os testes do sistema serão realizados em duas fases, na primeira fase os testes serão realizados com o auxílio de uma webcam para simular o ambiente de produção e após a finalização do desenvolvimento serão utilizadas as câmeras de segurança da Unifesp para a realização dos testes finais.

\subsection{Confecção da monografia final}

A última atividade será a confecção da Monografia Final (TCC-II). Essa monografia tem por objetivo apresentar o desenvolvimento completo do trabalho seguido de suas conclusões. Além disso, esta monografia deverá ser avaliada e julgada por uma comissão examinadora.


\section{Cronograma}
\label{sec:cronograma}

O desenvolvimento deste trabalho de conclusão de curso está dividido em cinco fases principais e na confecção de duas monografias. O plano de trabalho apresentado acima será executado conforme o cronograma apresentado na Tabela~\ref{tab:cronograma}.

\newcommand{\y}{\rule{25pt}{5pt}}
\newcommand{\x}{\hspace*{4pt}}
\setlength{\tabcolsep}{0pt}

\begin{table}
\centering
\begin{tabular}{|l|c|c|c|c|c|}
  \cline{2-6}
    \multicolumn{1}{l|}{} & \multicolumn{5}{c|}{\rotatebox{90}{2011\hspace{3pt}}}\\
  \cline{2-6}
  \multicolumn{1}{c|}{\textbf{Atividades}} &
                        \rotatebox{90}{Julho\hspace{3pt}} &
                        \rotatebox{90}{Agosto\hspace{3pt}} &
                        \rotatebox{90}{Setembro\hspace{3pt}} &
                        \rotatebox{90}{Outubro\hspace{3pt}} &
                        \rotatebox{90}{Novembro\hspace{3pt}} \\
  \hline
  1) Revisão Bibliográfica & \y & \y & \y & \y & \x \\
  \hline
  2) Implementação das técnicas selecionadas & \y & \y & \y & \x & \x \\
  \hline
  3) Testes do sistema & \x & \x & \y & \y & \x \\
  \hline
  4) Confecção da monografia final & \x & \x & \x & \y & \y \\
  \hline
\end{tabular}
 \caption{Cronograma}
 \label{tab:cronograma}
 \normalsize
\end{table}