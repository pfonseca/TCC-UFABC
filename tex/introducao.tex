%!TEX root = root.tex

\chapter{Introdução}
\label{cap:introducao}


\section{Contextualização}

A popularização dos \emph{smartphones} criou um mercado muito aquecido de venda de aplicativos e jogos através de lojas \emph{online} de aplicativos. Dentre estas lojas se destacam a loja \emph{Apple Store}~\cite{appstore} (iOS), \emph{Google Play}~\cite{googleplay} (Android) e \emph{Windows Store} (Windows Phone).


Permitindo que desenvolvedores ou pequenas empresas de desenvolvimento ( conhecidas como indies) publicassem seus aplicativos nas plataformas, o que já garantia de certa forma, mais um meio de divulgação de seus conteúdos.


Grandes empresas também estão investindo pesado no mercado de aplicativos para \emph{smartphones}, devido ao grande retorno financeiro nas vendas.



Além da venda de aplicativos, houve também o desenvolvimento de outras duas formas de faturamento, que são os banners de publicidade e vendas de itens dentro dos aplicativos, tática muito parecido com os antigos softwares \emph{freeware}, onde era possível utilizar o software de forma gratuita mas com algumas limitações, que seriam liberadas com o pagamento de uma taxa ou licença.






%%%%%%%--------

\section{Objetivos e Motivações}


Mapear as principais características do mercado de desenvolvimento de games para dispositivos móveis. 

Com o conhecimento das características do mercado aplicativos, um desenvolvedor será capaz de maximizar os ganhos, criando o jogo ou aplicativo para um público específico, utilizar um meio de rentabilidade mais eficaz, focar em uma determinada categoria de jogo, etc.
 

%%%%%------------------------

\section{Organização}


O Capítulo~\ref{cap:mercadoJogos} irá apresentar quais características dos mercados do Brasil e qual é no mundo, consolidando a venda de games para \emph{smartphones} e consoles. Neste 

%- Ascensão do mercado de games no Brasil (material do professor Isidro)
%- Ascensão do mercado de games no Mundo (material do professor Isidro)
%- Adicionar dados de vendas moveis (apps/games) IDG (material do professor Isidro)

3
O capítulo~\ref{cap:plataformas} 

Neste capítulo será tratado a venda de aparelhos móveis no mundo, pois este assunto influência diretamente as vendas de aplicativos.



- Adicionar dados de vendas Android/WindowsPhone/iOS
- Gráficos de distribuição de aparelhos
- Comparativo financeiro dos mercados moveis/PC/Plataforma.


4

- Tipos de jogos [Jogos casuais, jogos sociais, etc.
- Classificação Indicativa dos jogos (material do professor Isidro)


5

- Ambientes de desenvolvimento/linguagens/bibliotecas
- Principais ferramentas de desenvolvimento
- Tipos de monetização em games mobile


- Fontes de financiamento


7 

- Cases de sucesso (3 exemplos)
	- Como foi desenvolvido (ferramentas)
	- Onde foi desenvolvido
	- Estratégia de marketing
	- Fonte de investimento
	- Estatísticas de download ( se disponível )
	

