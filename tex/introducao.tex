%!TEX root = root.tex

\chapter{Introdução}
\label{cap:introducao}


\section{Contextualização}

A popularização dos \emph{smartphones} criou um mercado muito aquecido de venda de aplicativos e jogos através de lojas \emph{online} de aplicativos. Dentre estas lojas se destacam a loja \emph{Apple Store}~\cite{appstore} (iOS), \emph{Google Play}~\cite{googleplay} (Android) e \emph{Windows Store} (Windows Phone).


Permitindo que desenvolvedores ou pequenas empresas de desenvolvimento ( conhecidas como indies) publicassem seus aplicativos nas plataformas, o que já garantia de certa forma, mais um meio de divulgação de seus conteúdos.


Grandes empresas também estão investindo pesado no mercado de aplicativos para \emph{smartphones}, devido a quantidade de usuários e ao grande retorno financeiro nas vendas.



Além da venda de aplicativos, houve também o desenvolvimento de outras duas formas de faturamento, que são os banners de publicidade e vendas de itens dentro dos aplicativos, tática muito parecido com os antigos softwares \emph{freeware}, onde era possível utilizar o software de forma gratuita mas com algumas limitações, que seriam liberadas com o pagamento de uma taxa ou licença.






%%%%%%%--------

\section{Objetivos e Motivações}

Este trabalho tem como objetivo mapear as principais características do mercado de desenvolvimento de games para dispositivos móveis. 

Com o conhecimento das características do mercado aplicativos, um desenvolvedor será capaz de maximizar os ganhos, criando o jogo ou aplicativo para um público específico, utilizar um meio de rentabilidade mais eficaz, focar em uma determinada categoria de jogo, etc.
 

%%%%%------------------------

\section{Organização}


O Capítulo~\ref{cap:mercadoJogos} irá apresentar quais características dos mercados do Brasil e qual é no mundo, consolidando a venda de games para \emph{smartphones} e consoles. Neste 

%- Ascensão do mercado de games no Brasil (material do professor Isidro)
%- Ascensão do mercado de games no Mundo (material do professor Isidro)
%- Adicionar dados de vendas moveis (apps/games) IDG (material do professor Isidro)

O capítulo~\ref{cap:plataformas} 

Neste capítulo será tratado a venda de aparelhos móveis no mundo, pois este assunto influência diretamente as vendas de aplicativos.



% - Adicionar dados de vendas Android/WindowsPhone/iOS
% - Gráficos de distribuição de aparelhos
% - Comparativo financeiro dos mercados moveis/PC/Plataforma.



No capítulo~\ref{cap:classificacaoJogos} será tratado os tipos de jogos existentes e quais a principais características.
Neste capítulo será tratado a classificação indicativa dos jogos, quais são as categorias utlizadas no Brasil e no mundo. Também será tratado o conhecimento dos usuários sobre a utilização da classificação indicativa correta, no ponto de vista dos jogadores e dos responsáveis pelos jogadores.

% - Tipos de jogos [Jogos casuais, jogos sociais, etc.
% - Classificação Indicativa dos jogos (material do professor Isidro)


No capítulo~\ref{cap:desenvolvimentoFerramentas} serão abordados os ambientes de desenvolvimento mais utilizados bem como as linguagens, ferramentas e \emph{frameworks}, levando em consideração sua representatividade no mercado.

Será detalhado quais são os tipos de monetização em jogos móveis, visando identificar quais são as formas mais rentáveis para cada tipo de jogo. Quais são os pré-requisitos necessários para sua utilização e suas principais características.

Também será listado as fontes de financimaento de aplicativos mais utilizadas pelos desenvolvedores e empresas da área de games, como por exemplo, o Kickstarter\cite{kickstarter} e o site brasileiro Catarse\cite{catarse}. E algumas fontes privadas de investimentos, como bancos e programas de incentivos do governo.


% - Ambientes de desenvolvimento/linguagens/bibliotecas
% - Principais ferramentas de desenvolvimento
% - Tipos de monetização em games mobile


% - Fontes de financiamento


O capítulo~\ref{cap:caseSucesso} apresentará as estratégias de três casos de sucesso de jogos, abordando como e onde foi desenvolvido, estratégias de \emph{marketing} utilizadas, fonte de investimento para o desenvolvimento e as estatísticas de \emph{download}.\newline

Jogos analisados:

\begin{itemize}

	\item [Angry Birds~\cite{angrybirds}] \hfill \\
		Escolhido devido ao sucesso no início da popularização dos \emph{smartphones}, onde muitas estratégias de venda e divulgação ainda não existiam. Considerando que a concorrência era menor e o número de usuários de \emph{smartphones} estava em crescimento.

	\item [Flappy Bird~\cite{flappybird}] \hfill \\
		Escolhido devido ao sucesso repentino de um aplicativo/desenvolvedor desconhecido, utilizando gráficos e uma jogabilidade muito simples, atraindo público rapidamente. Mesmo após retirado das lojas de aplicativos, inúmeros clones surgiram.

	\item [Minecraft - Pocket Edition~\cite{minecraft}] \hfill \\
		Escolhido devido ao 
	
	

\end{itemize}


