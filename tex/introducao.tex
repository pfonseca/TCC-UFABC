%!TEX root = root.tex

\chapter{Introdução}
\label{cap:introducao}


\section{Contextualização}

A popularização dos \emph{smartphones} criou um mercado muito aquecido de venda de aplicativos e jogos através de lojas \emph{online} de venda de aplicativos. Dentre estas lojas se destacam a loja Apple Store (iOS), Google Play (Android) e Windows Store (Windows Phone).


Permitindo que desenvolvedores ou pequenas empresas de desenvolvimento ( conhecidas como indies) publicassem seus aplicativos nas plataformas, o que já garantia de certa forma, mais um meio de divulgação de seus conteúdos.


Grandes empresas também estão investindo pesado no mercado de aplicativos para \emph{smartphones}, devido ao grande retorno financeiro nas vendas.



Além da venda de aplicativos, houve também o desenvolvimento de outras duas formas de faturamento, que são os banners de publicidade e vendas de itens dentro dos aplicativos, tática muito parecido com os antigos softwares \emph{freeware}, onde era possível utilizar o software de forma gratuita mas com algumas limitações, que seriam liberadas com o pagamento de uma taxa ou licença.








%%%%%%%--------

\section{Objetivos e Motivações}


Mapear as principais características do mercado de desenvolvimento de games para dispositivos móveis. TODO
 

%%%%%------------------------

\section{Organização}


