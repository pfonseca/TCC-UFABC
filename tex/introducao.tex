%!TEX root = root.tex

\chapter{Introdução}
\label{cap:introducao}


\section{Contextualização}

Com popularização dos \emph{smartphones} criou-se um mercado muito aquecido de venda de aplicativos e jogos através de lojas \emph{online} de aplicativos. Dentre estas lojas se destacam as lojas da \emph{Apple Store}~\cite{appstore} (iOS), \emph{Google Play}~\cite{googleplay} (Android) e \emph{Windows Store}~\cite{windowsphone} (Windows Phone). Permitindo que desenvolvedores ou pequenas empresas de desenvolvimento ( conhecidas como \emph{indies} ) publicassem seus aplicativos nas plataformas, o que garante mais um meio de divulgação de seus conteúdos.


Grandes empresas também estão investindo pesado no mercado de aplicativos para \emph{smartphones}, devido a quantidade de usuários e ao grande retorno financeiro nas vendas.



Além da venda de aplicativos, houve também a criação de outras duas formas de faturamento, que são os baners de publicidade e vendas de itens dentro dos aplicativos, tática muito parecido com os antigos softwares \emph{freeware}, onde era possível utilizar o software de forma gratuita mas com algumas limitações, que seriam liberadas com o pagamento de uma taxa ou licença.






%%%%%%%--------

\section{Objetivos e Motivações}

Este trabalho têm como objetivo mapear as principais características do mercado de desenvolvimento de jogos para dispositivos móveis. Identificando os principais padrões desse mercado e de seus usuários.

Com o conhecimento das características do mercado aplicativos, um desenvolvedor será capaz de maximizar os lucros, criar o jogo ou aplicativo para um público específico, utilizar um meio de rentabilidade mais eficaz, focar em uma determinada categoria de jogo, etc.
 

%%%%%------------------------

\section{Organização}


O Capítulo~\ref{cap:mercadoJogos} irá apresentar as características do mercado de jogos no Brasil e no mundo, consolidando a evolução do faturamento nas vendas de games para \emph{smartphones} e consoles. Também será tratado o perfil dos jogadores e as principais categorias de jogos.

Outro ponto abordado neste capítulo será a revolução iniciada pelas lojas de aplicativos, apresentando suas principais características e realizando um comparativo entre as principais lojas existentes.



%- Ascensão do mercado de games no Brasil (material do professor Isidro)
%- Ascensão do mercado de games no Mundo (material do professor Isidro)
%- Adicionar dados de vendas moveis (apps/games) IDG (material do professor Isidro)

O capítulo~\ref{cap:plataformas} apresentará a venda de aparelhos móveis no mundo. Este assunto influência diretamente nas vendas de aplicativos, pois quanto maior a base de aparelhos de uma determinada plataforma, maiores serão as chances de retorno financeiro nesta plataforma. Está característica deve ser levada em consideração na fase do projeto de seleção das plataformas que o jogo irá suportar.



% - Adicionar dados de vendas Android/WindowsPhone/iOS
% - Gráficos de distribuição de aparelhos
% - Comparativo financeiro dos mercados moveis/PC/Plataforma.



No capítulo~\ref{cap:classificacaoJogos} será tratado os tipos de jogos existentes e quais a principais características.
Neste capítulo será tratado a classificação indicativa dos jogos, quais são as categorias utlizadas no Brasil e no mundo. Também será tratado o conhecimento dos usuários sobre a utilização da classificação indicativa correta, no ponto de vista dos jogadores e dos responsáveis pelos jogadores.

% - Tipos de jogos [Jogos casuais, jogos sociais, etc.
% - Classificação Indicativa dos jogos (material do professor Isidro)


No capítulo~\ref{cap:desenvolvimentoFerramentas} serão abordados os ambientes de desenvolvimento mais utilizados, bem como as linguagens de desenvolvimento, ferramentas e \emph{frameworks}, levando em consideração sua representatividade no mercado. Serão tratados ferramentas específicas para cada plataforma e ferramentas híbridas que são independentes de plataforma.


Será detalhado quais são os tipos de monetização em jogos móveis, visando identificar quais são as formas mais rentáveis para cada tipo de jogo. Quais são os pré-requisitos necessários para sua utilização e suas principais características.


Também será listado as fontes de financiamento de aplicativos mais utilizadas pelos desenvolvedores e empresas da área de games, como por exemplo, o Kickstarter\cite{kickstarter} e o site brasileiro Catarse\cite{catarse}. E algumas fontes privadas de investimentos, como bancos e programas de incentivos do governo.


% - Ambientes de desenvolvimento/linguagens/bibliotecas
% - Principais ferramentas de desenvolvimento
% - Tipos de monetização em games mobile


% - Fontes de financiamento


O capítulo~\ref{cap:caseSucesso} apresentará as estratégias de três casos de sucesso de jogos para \emph{smartphones}, abordando como e onde foi desenvolvido, estratégias de \emph{marketing} utilizadas, fonte de investimento para o desenvolvimento e as estatísticas de \emph{download} e vendas.\newline

Jogos analisados:

\begin{itemize}

	\item [Angry Birds~\cite{angrybirds}] \hfill \\
		Escolhido devido ao sucesso antes da popularização dos \emph{smartphones}, onde muitas estratégias de venda e divulgação ainda não existiam. Considerando que a concorrência era menor e o número de usuários de \emph{smartphones} estava em crescimento.

	\item [Flappy Bird~\cite{flappybird}] \hfill \\
		Escolhido devido ao sucesso repentino de um aplicativo/desenvolvedor desconhecido, utilizando gráficos e uma jogabilidade muito simples, atraindo público rapidamente. Mesmo após retirado das lojas de aplicativos, inúmeros clones surgiram.

	\item [Minecraft - Pocket Edition~\cite{minecraft}] \hfill \\
		Escolhido por manter o sucesso desde seu lançamento, que ocorreu em 2009 para computador e portado para dispositivos móveis em 2011. Devido ao seu sucesso em todas as plataformas, foi adquirido pela Microsoft em 2014, por 2,5 bilhões de dólares~\cite{minecraft-bought}.
	
	

\end{itemize}





No capítulo~\ref{cap:conclusao}, será apresentado uma conclusão das informações levantadas neste trabalho, visando destacar as principais características e identificando padrões, auxiliando os desenvolvedores e usuários na decisão da plataforma atendida pelo jogo a ser desenvolvido e pela decisão da melhor forma de retorno financeiro baseado nas características identificadas.



\section{Metodologia}

Este trabalho será baseado nos resultados de pesquisas de mercado de inúmeras fontes confiáveis de pesquisa, visando cruzar informações para identificar caraterísticas no desenvolvimento de jogos movéis. Todas as fontes utilizadas são materiais públicos e serão referenciados neste trabalho.\newline
O trabalho irá mostrar o mercado de jogos como um todo, mas terá um foco especial no mercado de \emph{smartphones}. O mercado de aplicativos é recente, sendo impulsionado pelo lançamento do loja de aplicativos para \emph{iPhone} em 2008.



Os materiais utilizados para análise serão retirados das seguintes fontes:


\begin{itemize}

	\item [\textbf{IDG}~\cite{idg}] Grupo americano de pesquisas e investimentos de capitais de risco (\emph{venture capital}), responsável pelo levantamento de vendas de jogos e a criação de projeções de vendas para os próximos anos.

	\item [\textbf{BNDES}~\cite{bndes}] Banco Nacional de Desenvolvimento, responsável pelo investimentos em todos os segmentos da economia, de âmbito social, regional e ambiental. Que levantou uma série de informações, em conjunto com a USP, relacionado ao mercado de jogos no brasil.
	
	\item [\textbf{NewZoo}~\cite{newzoo}] Empresa especializada em pesquisa de mercado de jogos. Criando relatórios de análise do mercado por país e tendência de vendas por segmento de jogos. Possui grandes clientes do setor de jogos, como por exemplo: EA~\cite{ea}, Valve~\cite{valve}, Square Enix~\cite{square-enix}, entre outros.
	
	
	\item[\textbf{SIOUX/Blend New Research}~\cite{sioux}~\cite{blend}] Duas empresas de pesquisa de mercados que, em conjunto, produziram um excelente trabalho de mapeamento de mercado de desenvolvimento de jogos.
	
	\item [Entre outros.]

\end{itemize}





% Descrição: Apresentar o método adotado no desenvolvimento científico do tema: entrevista, questionário, busca, observação, experimentação, testes e ensaios, simulação, implementação (requisitos, desenvolvimento, e testes), etc. Especificar o universo da pesquisa: características e quantificação, tamanho da amostra, etc. Discutir a viabilidade e a validade da metodologia adotada. Descrever e especificar os elementos que compõem a metodologia adotada, ou seja, os componentes do: desenvolvimento, pesquisa, experimento, simulação, testes, etc. Esta descrição deverá conter informações que possibilitem a reprodução do método: condições experimentais, equipamentos e instrumentos utilizados, softwares utilizados (incluindo o ambiente de execução e a versão), processos utilizados (qualificação e certificação), e pessoal técnico (quantidade e qualificação), entre outros. Os dados de entrada, parâmetros e configurações também devem ser especificados, tabulados e apresentados


